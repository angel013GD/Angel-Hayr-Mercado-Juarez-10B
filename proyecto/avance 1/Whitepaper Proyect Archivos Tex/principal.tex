\documentclass{report}
\usepackage[utf8]{inputenc}

% Títulos automáticos en español
\usepackage[english]{babel}

% Soporte para buenas urls e hipervínculos entre secciones
\usepackage{hyperref}

% Citas y referencias en formato APA
% Si quiere las citas y referencias en IEEE comente esta línea
\usepackage{apacite}

% Imágenes y figuras
\usepackage{graphicx}

% Código fuente con números de línea
\usepackage{listings}
% Puede cambiar el lenguaje de código fuente
% https://www.overleaf.com/learn/latex/code_listing#Supported_languages
\lstset{
    language=C,
    basicstyle=\footnotesize,
    numbers=left,
    stepnumber=1,
    showstringspaces=false,
    tabsize=1,
    breaklines=true,
    breakatwhitespace=false,
}


\def \unidad{Universidad Tecnológica de Tijuana }
\def \programa{Ingeniería En Desarrollo Y Gestión De Software }
\def \curso{Aplicaciones Web Progresivas}
\def \titulo{Project proposal}

\def \autores{
    Garcia Gonzales Christian Andres\\
    Lopez Bautista Cristian Alexis\\
    Mercado Juarez Angel Hayr\\
    Salas Diaz Guillermo\\
    Santillan Galaviz Ken Antonio\\
    
    \vspace{0.5cm}
    
    Teacher \\
    Dr. Ray Brunett Parra\\
}
\def \fecha{02 February 2024}
\def \lugar{
    Tijuana B.C, 
   México
}

% Inicia el documento 
\begin{document}

% Inserta la portada del documento
\input{portada}

\tableofcontents

\chapter{Project proposal}\label{Project proposal}
\section{Introduction}\label{intro}
In the current context, the efficient management of incidents related to the safety and health of employees has become a priority for organizations. In this regard, our project is oriented towards developing a robust web application that facilitates this process effectively. The application is designed to cater to both administrative and production personnel, and its accessibility is key to its success. 

\section{Why Use PWA?}\label{PWA}

The project consists of a web application that allows any organization to efficiently and effectively manage incidents related to the safety and health of its employees. The platform can register and monitor incidents that affect the safety and well-being of employees.
\vspace{0.5cm}
Progressive Web Applications (PWA) are computer programs that use web technologies such as HTML, CSS, JS, and PHP. These tools allow access to the application through a web browser and facilitate the creation and management of online content.
\vspace{0.5cm}
In the current technological era, mobile implementation in application development has become a crucial pillar for providing immersive and efficient digital experiences. In this context, Progressive Web Applications (PWA) have emerged as a key approach to mobile app development.
\vspace{0.5cm}
Our application is intended for use by both administrative and production staff, making it essential for it to adapt to the majority of devices used by this personnel.
\vspace{0.5cm}
Our goal is to employ current web technologies to enhance the user experience, aligning with the central concept of PWAs: creating a reliable, fast, and attractive application, regardless of the browser or device used.
\vspace{0.5cm}
By using a PWA, we can increase response efficiency in the event of an incident in the workplace. The flexibility of PWAs allows all personnel to use the application, such as nursing staff efficiently updating data, thus reducing the response time of medical personnel.
\vspace{0.5cm}
As PWAs can leverage device hardware, users can use their device's camera to quickly scan QR codes, facilitating the identification of employees in emergency situations and providing access to emergency personal information, improving specialized medical care.
\vspace{0.5cm}
User interaction is the primary focus of implementing PWA in our project, aiming for each user to have quick, efficient, and seamless access within the company. This is crucial for a service provider like us.
\vspace{0.5cm}
Through the resources offered by PWAs, we can access the application from any device, creating an engaging environment through a simple yet effective means. This is essential for incident control, as it facilitates access to important employee data in case of issues, regardless of the diversity of devices they may have.
\vspace{0.5cm}
Moreover, being a PWA, updates are constant and automatic, providing better adaptability without requiring manual updates from the personnel.
\vspace{0.5cm}
Our application offers various functions that are maximized across different devices, from phones and tablets to computers and large screens like TVs.
\vspace{0.5cm}

\section{Technological Context}\label{Technological Context}

The application is situated in a technological environment where mobility and accessibility are fundamental. Web applications offer an efficient solution, and particularly, Progressive Web Apps (PWAs) have emerged as a comprehensive response to the current demands of mobile app development. Their ability to function on any browser and device provides essential flexibility for our project.

\section{Adaptability to Diverse Devices}\label{Adaptability to Diverse Devices}
In the workplace, the diversity of devices used by personnel is a reality that cannot be overlooked. From mobile phones and tablets to computers and even television screens, our application must adapt to each of these contexts. The choice of a PWA is fully justified by providing a seamless user experience, regardless of the device used.


\section{Efficiency in Incident Response}\label{Efficiency in Incident Response}
The nature of our application, focused on incident management, demands a quick and effective response to critical situations. PWAs offer optimal performance, eliminating common loading times in native applications. This translates to a significant increase in response efficiency to unexpected events in the work environment.

\section{Interaction and Accessibility}\label{Interaction and Accessibility}
Effective interaction with our users is a priority. The implementation of a PWA ensures fast, efficient, and fluid access for all employees, regardless of their familiarity with technology. The inclusion of a QR code to access the application further simplifies the process, especially in emergency situations, allowing users to obtain critical information easily.

\section{Notification and Collaboration}\label{Notification and Collaboration}
The ability to notify emergency contacts and collaborate effectively in real-time is essential in incident management. PWAs enable smooth communication and real-time updates, enhancing collaboration between administrative, production, and medical staff. This directly contributes to a quicker and coordinated response to any critical situation.

\section{Automated Updates}\label{Automated Updates}
Efficient application management involves staying up-to-date with the latest features and security fixes. The nature of PWAs allows automatic updates, eliminating the need for manual updates by the staff. This ensures that the application is always up-to-date, improving its adaptability as the company's needs evolve.
\section{Functional Versatility}\label{Functional Versatility}
Given the diversity of functions that our application offers, it is essential to make the most of them in various contexts. From quick data capture by production personnel to detailed review by medical staff on larger devices, the versatility of a PWA becomes an invaluable asset.

\section{Conclusion}\label{Conclusion}
In conclusion, choosing a Progressive Web App for our incident management application is a strategic step. It provides a solid technological foundation to ensure efficiency, accessibility, and versatility at all levels of the organization. By embracing PWAs, we are committed to providing a reliable and attractive user experience, regardless of the device or browser used. This approach not only optimizes incident management but also contributes to the development of an effective safety and health culture in the workplace.

\end{document}
